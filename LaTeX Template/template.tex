\documentclass[sigsmall]{acmart}

\title{A Survey of LoRa Networking Techniques and Interference Mitigation Testing}
\author{Alexandria Macko, Justin Verhosek, Joshua Buxton}

\setcopyright{none}
\settopmatter{printacmref=false}

\begin{document}
\maketitle
\section*{Abstract}
LoRa, a popular wireless option throughout the world, is a favorable choice for providing network coverage to IoT devices given its open-source and low-cost nature. Networking techniques are effective in utilizing LoRa. The four layers, which include the physical, link, MAC, and App layers, all have adjustments that can be made to better implement LoRa in different environments. Areas where LoRa as seen advancements is the utililiazation of DeepLoRa and NeLoRa, which each aim to improve LoRa usage and communication by via modification of a number of factors. Different environmental elements do have a big impact on LoRa usage, as foliage, wind, humidity, rainfall, temperature, etc. A useful tool in mitigating these factors is adjusting the Spread Factor of the data transfer. Along with outdoor concerns, new movement has been taken to implement LoRa indoors using a new topology that implements constructive interference called LoRaIN, which manages to improve LoRa’s efficiency from 62\% up to 92\%. Another new system being developed is Sen-Fence a virtual fencing technique that is able to narrow down the focus of LoRa to a specific geographical area, reducing the interference caused by other wireless signals that would normally be in range. Additionally, SRLoRa is another method to improve weak signal strength via the use of multiple gateways that has been proven significantly efficient. This comprehensive survey of the current field of LoRa provides a look at both the network techniques and recent developments in interference mitigation.Looking into the future further research can be done to the utilization and implementation of virtual fencing, LoRaIN’s constructive interference technique, and the scalability of SRLoRa \cite{hopper:compilers101}.

\section*{Introduction}
LoRa has been increasingly used as a means of providing connectivity to IoT devices all over the world. LoRa, or long-range low-power, belongs to a classification of networks called Low-Power Wide-Area Networks, also known as LPWANS. As opposed to other LPWANS, LoRa is gaining popularity due to its open-source low-cost nature. The survey begins by introducing the fundamental principles of LoRa technology, highlighting its strengths in achieving exceptional coverage, deep indoor penetration, and energy-efficient operation. We explore the physical layer components, modulation schemes, and spread spectrum techniques that make LoRa an attractive choice for IoT deployments.Before jumping into the research surrounding LoRa, we must discuss the LoRa networking stack layers and the configuration settings that impact the performance of LoRa. Since LoRa is open-source, the user has the ability to configure and customize the network settings that influence the network performance, data transmission rate, and signal spread. These settings must be considered in interference testing as they could sway our interference data, measured in RSSI or, Received Signal Strength Indication [citatation]. In this survey we discuss the LoRa network stack, provide an analysis, discuss forms of environmental interference and indoor usage, combating wireless interference, and weak signal decoding.

\section*{LoRa Network Stack}
The survey paper “LoRa Networking Techniques for Large-scale and Long-term IoT: A down-to-top Survey” by Chenning Li and Zhichao Cao provides significant detail into the various layers of the LoRa networking stack. Li and Cao aim to categorize and compare LoRa networking techniques, referencing numerous studies that utilize configuration settings provided in each layer of the network stack to achieve end goals such as reduced interference or increased signal range, often with the side effect of increased power consumption. The LoRa networking stack is broken up into four layers: the Physical Layer (PHY), the Link Layer (or the Hardware Abstraction Layer, HAL), the MAC layer, and the App Layer.  To understand the research surrounding LoRA, one must understand the architecture of the LoRa networking stack, starting with the PHY layer.  In the PHY layer, end nodes modulate and encode data, while gateways demodulate and decode data. The PHY layer utilizes Chirp Spread Spectrum, or CSS, which uses chirps or pulses to communicate data over a certain frequency for modulation and demodulation. Next, the Link Layer consists of four abstracted settings that are used to control certain performance behaviors of LoRa: Spread Factor, BW, TP, and Channel Frequency. Spread Factor, configurable from the end nodes, ranges from 7-12 and configures the communication range at the expense of energy consumption. The higher the Spread Factor, the more time the data lingers, which reduces the data rate but also improves the signal's ability to handle interference. The next setting, BW (bandwidth), configures the data transmission rate; the lower the data rate, the more resilient the data is to outside noise and interference. The TP value extends the range of the signal with the consequence of increased energy consumption. The last configuration, Channel Frequency, offers different communication ranges by the ability to select different channels. Moving on to the MAC Layer, which assists in achieving efficient data delivery by regulating the transmission of each end node. Overall, the MAC Layer seeks to regulate power management and collision avoidance. Power management in the MAC Layer is divided into three classifications: Class A, where an end node only consumes power when it’s transmitting, Class B where an end node periodically transmits on a schedule, and Class C where an end node is always transmitting data. Lastly, the App Layer secures data delivery and utilizes authentication and encryption to promote data security. Understanding the layered nature of LoRa is crucial when it comes to interference testing as each layer provides configuration options that can sway the resulting data.[citation]

\section*{LoRa Analysis}
Recent Advances in LoRa: A Comprehensive Survey begins with a basic rundown of LoRa, its key factors such as large communication range and low energy consumption, and how well, as a modulation technique, it works. The key parameters of LoRa discussed are spreading factor (SF), bandwidth (BW), code rate (CR), carrier frequency (CF) and transmission power (TP). The survey is broken down into 4 categories: LoRa Analysis, LoRa Communication, LoRa Security, and LoRa-enabled Applications.
LoRa analysis focused on performance measurements, analytical models, simulators and testbeds. Some of the important findings were the line-of-sight and non-line-of-sight ranges, 10km and 2km respectively, testing how factors’ interference greatly influences path loss, path loss models such as DeepLoRa, a deep learning-based path loss estimation framework, focused around land-covers, energy harvesting using ambient RF, the ability to test LoRa networks using simulators, and testbeds, which can be similar to simulators, but rely on replicating the behavior of physical sensors rather than generated data.
LoRa communication centered around modulation and demodulation, MAC protocols and configuration settings. LoRa modulation is based on the frequency shift of the chirp and many modified methods aim to increase the data rate by having the LoRa symbols carry more information. In terms of demodulation, there are many new methods such as NeLoRa that aim to improve low-SNR decoding, to improve demodulation when working with very low SNR or RSSI. Some other important topics are payload recovery schemes and decreasing clock rates to lower energy consumption. Another key area is collision disambiguation, or the separation and correct decoding of multiple collided signals. MAC protocols can be divided into contention and schedule-based. Contention is mainly random access, nodes are listening for the shared medium. Schedule-based access the shared medium using deliberate time slots. A big takeaway from this survey was all the enhanced LoRa methods in terms of analysis and communication [citation].

\section*{Environmental Interference}
In a study titled “Factors that affect LoRa Propagation in Foliage Medium,” Rabeya Anzum discusses forms of environmental interference, specifically, how foliage and climate decreases range and signal quality. The study employs Fresnel zones, or elliptical-shaped areas between a transmitter and receiver, to accurately calculate where to place transmitters and receivers to measure foliage's interference. In this case, a transmitter and receiver were stationed on each side of a line of palm trees, with the foliage of the palm trees centered within the Fresnel zone. Other considerations of the study include wind (as the wind causes moving foliage), rainfall, humidity, temperature, and geolocation, all of which cause a variation in signal strength. The LoRa settings in the PHY layer configured for the study are as follows: Frequency: 868MHz, BW: 125 Hz, Spread Factor: SF7-SF10, Antenna Gain: 2dBi, Tx-power: 14dbm. The results of the study indicate that as the Spread Factor increases, so does packet loss, especially as the number of trees increases between the transmitter and receiver. An important takeaway is the influence of the Spread Factor on the quality of data transfer, and the impact weather has on getting consistent measurements of interference. As the study suggests, we should aim to gather data with low humidity, moderate temperature, and low wind. [citation]

\section*{Indoor Usage}
In Boosting Reliability and Energy-Efficiency in Indoor LoRa Mahbubur Rahman and Abusayeed Saifullah discuss LoRa capabilities indoors that are often overlooked in its usage. LoRa is not typically used indoors because of the shadowing effect, where a higher packet was loss than when placed outdoors, failure to retransmit the acknowledgment (ACK), and other issues related to medium to heavy networks. To combat this, Rahman and Saifullah developed LoRaIN (LoRa Indoor Network), a new approach to implementing LoRa indoors. LoRaIN implements constructive interference in order to get more consistent packet delivery. This method utilizes the carrier activity detection (CAD) feature of LoRa and also an unused octet of the LoRaWAN frame. It boosts the LoRa's ability to receive ACKs accurately, and through simulations the maximum allowable temporal displacement between two LoRa sensors was discovered. To test it LoRaIN was implemented on 20 Dragino LoRa Hat nodes running on a Rasberry Pi. After the testing was completed a significant improvement was seen as the efficiency of LoRa jumped from 62\% with LoRaWAN to 92\% using LoRaIN. Important to note that these mitigation techniques are completely novel, as indoor usage of LoRa is rarely supported and more specifically this is the first usage constructive interference in the context of LoRa. [citation]

\section*{Combating Wireless Interference}
In Combating Interference for Long Range LoRa Sensing, Binbin Xie and Jie Xiong discuss that with the increase in wireless sensing, a major issue for LoRa implementation is that most systems assume no outside wireless interference, which in practice is often incorrect. Their answer to this issue was to develop and test “Sen-fence,” which seeks to narrow down the area of focus of a LoRa system in order to mitigate surrounding interference. Rather than searching the whole surrounding area that LoRa is capable of and potentially picking up interference from other wireless signals, a predetermined target is set, and a virtual fence is created. They plan on this target being very specific, claiming that a virtual fence can focus on a 0.8 m x 0.8 m space to cover only a single chair. Through theoretical analysis and experiments, it was concluded that interference was significantly mitigated, even seeing that with this approach, the signal can reach up to 50 m, or twice the state of the art. One possible limitation that was not fully addressed was how this technique would combat interference in direct line of sight between the target at the receiver. Other than this one concern, Sen-Fence appears to be an extremely viable method in narrowing down the area of focus in LoRa communication to maximize its signal strength.  [citation]

\section*{Weak Signal Decoding}
SRLoRa: Neural-enhanced LoRa Weak Signal Decoding with Multi-gateway Super Resolution. SRLoRa is a deep neural network LoRa decoder that uses multiple gateways in order to decode weak LoRa signals (p. 270). Although signal-to-noise ratio (SNR) can be very low in decoding signals, with the addition of interference in environments, it can be lacking (p. 270). SRLoRa’s goal is to decode very weak signals by merging signals of the same chirp symbol received at multiple gateways (p. 271). Some of the problems SRLoRa solves are merging signals under low SNR, time misalignment, and geographical diversity. To solve low SNR merging, SRLoRa uses “...an interleaving DNN structure containing denoising layers based on a convolutional neural network (CNN) and merging layers” (p. 271). To combat time misalignment, extracted features from CNNs are used to merge instead of the raw signal. The CNN “learns” patterns that allow it to permit time misalignment (p. 271). Max and Min operators are used to merge the signals based on important features of each signal. The authors did an indoor test in an office space and an outdoor test in parking lots across their campus. They used 4 LoRa gateways and 1 LoRa node and tested SNR levels from -35 to -15 dB and SF values of 8, 9 and 10. Symbol error rate (SER) and SNR were the two performance metrics. They were compared to 5 different methods: Dechirp, Charm, NELoRa, NELoRaV and NECharm. The results show that at every SF value, SRLoRa had lower SER than the other methods and the most SNR gain, averaging 4.67 dB for each SF. This also increases the coverage area 1.84x that of dechirp. There was also testing using different numbers of gateways. The results show that the higher the number of gateways, the higher the SNR gain [citation]. I think an important takeaway is that SRLoRa has proven itself to efficiently and significantly improve weak signal decoding using multiple gateways. Some limitations are the number of gateways and the environments tested. Neither the indoor or outdoor tests seemed to extensively test interference.

\section*{Conclusion}
LoRa, as a long range, low energy networking technique, is growing more prominent in the IoT world. In this paper, we have discussed the LoRa network stack’s physical, link, MAC, and App layers, LoRa evaluation and communication advances, assessed LoRa in different environments seeing the importance of spread factor, and new implementations to utilize LoRa indoors, combat wireless interference and improve weak signal decoding. All of these sections highlight the growth and potential growth LoRa holds for the future. Looking towards further research, virtual fencing techniques can be implemented into other forms of wireless communication beyond the scope of LoRa. Along with this the topology of LoRaIN, particularly, constructive interference can be further explored to utilize it for increasing the efficiency of LoRa in other environments. A further work direction for SRLoRa would be larger scalability testing such as adding more nodes or increasing the actual size of the test range and harsher environmental or interference factors. These are just a select few possibilities for further research into these advancements of LoRa, as supplied by the papers surveyed. As LoRa continues to play an ever-expanding role in the IoT ecosystem, the findings presented here consolidate networking techniques, configuration settings, and contribute to the knowledge base necessary for the successful deployment and management of interference mitigation in LoRa-based IoT networks.

\bibliographystyle{acm}
\bibliography{myBib}
\end{document}